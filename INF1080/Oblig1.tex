% Kompiler med "pdflatex latex.tex"
% LaTeX på Wikibooks: http://en.wikibooks.org/wiki/LaTeX/
% Finn symboler ved å tegne: http://detexify.kirelabs.org/classify.html

\documentclass[norsk,a4paper]{article}

\usepackage[T1]{fontenc}
\usepackage[utf8]{inputenc}
\usepackage[norsk]{babel}
\usepackage{amsmath}


\title{Innlevering 1 INF1080}
\author{Endre Wullum\\ \texttt{endrewu@ulrik.uio.no}}

\begin{document}

\maketitle

\section*{Oppgave 1.8}

\begin{description}
\item[(a)] Sann
\item[(b)] Usann
\item[(c)] Sann
\item[(d)] Sann
\item[(e)] Usann
\item[(f)] Usann
\item[(g)] Sann
\item[(h)] Sann
\end{description}

\section*{Oppgave 1.9}

\begin{description}
\item[(a)] \{5, 7, 9\}
\item[(b)] \{0, 2, 4\}
\item[(c)] \{5, 7, 9\}
\item[(d)] \{6, 8\}
\item[(e)] $\emptyset$ 
\item[(f)] \{0, 2, 4, 6, 8\}
\end{description}

\section*{Oppgave 1.10}

\begin{description}
\item[(a)] 16 delmengder
\item[(b)] 32 delmengder
\end{description}

%Her er noen flere uttrykk:
%\begin{align*}
%	P &= Q \cup S \\
%	S &= Q \cap R \\
%	S &\subseteq R \\
%	R &= F \setminus R \\ 
%	R &= \emptyset
%\end{align*}

%Her er et uttrykk: $P = \{a, b, c \}$.

\end{document}

