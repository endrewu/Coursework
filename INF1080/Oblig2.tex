% Kompiler med "pdflatex latex.tex"
% LaTeX på Wikibooks: http://en.wikibooks.org/wiki/LaTeX/
% Finn symboler ved å tegne: http://detexify.kirelabs.org/classify.html

\documentclass[12pt,norsk,a4paper]{article}

\usepackage[T1]{fontenc}
\usepackage[utf8]{inputenc}
\usepackage[norsk]{babel}
\usepackage{amsmath}
\usepackage{parskip}


\title{Innlevering 2 INF1080}
\author{Endre Wullum\\ \texttt{endrewu@ulrik.uio.no}}

\begin{document}

\maketitle

\section*{Oppgave 3.8}

\begin{description}
\item[(a)]
\begin{enumerate}
\item $(M \land H) \rightarrow \lnot B$
\item $\lnot(B \lor H) \land M $
\end{enumerate}
\item[(b)]
\begin{enumerate}
\item Hvis det er slik at jeg drikker melk, hvis jeg spiser honning, så spiser jeg brød.
\item Hvis det er slik at jeg spiser honning, så spiser jeg brød hvis jeg drikker melk.
\end{enumerate}
\item[(c)] Dersom man setter sannhetsverdiene til H, M og B til henholdsvis 0, 1 og 0 eller 0, 0 og 0 vil formelen $(H \rightarrow M) \rightarrow B$ bli usann og formelen $H \rightarrow (M \rightarrow B)$ bli sann.
\item[(d)] Med utgangspunkt i sannhetsverditabellen vil man finne at F er ekvivalent med formelen $(H \land (M \lor B)) \lor (M \land B)$

\end{description}

%Her er noen flere uttrykk:
%\begin{align*}
%	P &= Q \cup S \\
%	S &= Q \cap R \\
%	S &\subseteq R \\
%	R &= F \setminus R \\ 
%	R &= \emptyset
%\end{align*}

%Her er et uttrykk: $P = \{a, b, c \}$.

\end{document}

