% Kompiler med "pdflatex latex.tex"
% LaTeX på Wikibooks: http://en.wikibooks.org/wiki/LaTeX/
% Finn symboler ved å tegne: http://detexify.kirelabs.org/classify.html

\documentclass[12pt,norsk,a4paper]{article}

\usepackage[T1]{fontenc}
\usepackage[utf8]{inputenc}
\usepackage[norsk]{babel}
\usepackage{amsmath}
\usepackage{parskip}


\title{Innlevering 3 INF1080}
\author{Endre Wullum\\ \texttt{endrewu@ulrik.uio.no}}

\begin{document}

\maketitle

\section*{Oppgave 4.13}
Følgende grupper av formler er ekvivalente:
\begin{enumerate}
\item $\neg (\neg A \lor B)$, $\neg (A \rightarrow B)$, $(A \land \neg B)$
\item $\neg (\neg A \lor \neg B)$, $\neg (A \rightarrow \neg B)$, $(A \land B)$
\item $\neg (\neg A \rightarrow B)$, $(\neg A \land \neg B)$, $\neg (A \lor B)$
\item $\neg (\neg A \rightarrow \neg B)$, $(\neg A \land B)$, $\neg (A \lor \neg B)$
\end{enumerate}

\section*{Oppgave 5.3}
\begin{description}
\item[(a)]Er hverken tautologi eller kontradiksjon ettersom enhver valuasjon som gjør P sann vil gjøre formelen usann, men enhver valuasjon som gjør P usann vil gjøre formelen sann. Altså er den både oppfyllbar og falsifiserbar.
\item[(b)]Er en tautologi ettersom det ikke finnes noen valuasjon som kan falsifisere den.
\item[(c)]Er hverken en tautologi eller en kontradiksjon ettersom enhver valuasjon som gjør P usann vil gjøre formelen sann, men en valuasjon som gjør P sann og Q sann vil gjøre formelen usann. Altså er den både oppfyllbar og falsifiserbar.
\item[(d)]Er en tautologi ettersom det ikke finnes noen valuasjon som kan falsifisere den.
\item[(e)]Er hverken en tautologi eller en kontradiksjon ettersom en valuasjon som gjør P usann, Q usann og R usann vil gjøre formelen sann. En valuasjon som gjør P sann vil gjøre formelen usann.
\item[(f)]Er en kontradiksjon, ettersom det ikke finnes en valuasjon som kan gjøre formelen oppfyllbar
\end{description}

\section*{Oppgave 5.8}
En formel med konnektivene $\land, \lor, \neg og \rightarrow$ som er ekvivalent med den gitte sannhetsverditabellen for $(F \oplus G)$ kan for eksempel være $(F \land \neg G) \lor (\neg F \land G)$.

Vi kan se at denne formelen også får følgende sannhetsverditabell:

\begin{tabular}{c|c|c}
  $F$ & $G$ & $(F \land \neg G) \lor (\neg F \land G)$ \\
  \hline
  $1$ & $1$ & $0$ \\
  $1$ & $0$ & $1$ \\
  $0$ & $1$ & $1$ \\
  $0$ & $0$ & $0$ \\
\end{tabular}

%Her er noen flere uttrykk:
%\begin{align*}
%	P &= Q \cup S \\
%	S &= Q \cap R \\
%	S &\subseteq R \\
%	R &= F \setminus R \\ 
%	R &= \emptyset
%\end{align*}

%Her er et uttrykk: $P = \{a, b, c \}$.

\end{document}

