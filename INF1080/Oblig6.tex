% Kompiler med "pdflatex latex.tex"
% LaTeX på Wikibooks: http://en.wikibooks.org/wiki/LaTeX/
% Finn symboler ved å tegne: http://detexify.kirelabs.org/classify.html

\documentclass[12pt,norsk,a4paper]{article}

\usepackage[T1]{fontenc}
\usepackage[utf8]{inputenc}
\usepackage[norsk]{babel}
\usepackage{amsmath}
\usepackage{parskip}
\usepackage{fixltx2e}

\title{Innlevering 6 INF1080}
\author{Endre Wullum\\ \texttt{endrewu@ulrik.uio.no}}

\begin{document}

\maketitle

\section*{Oppgave 9.1}
\begin{description}
\item[(a)]$f$ og $h$ er injektive
\item[(b)]$f$ og $i$ er surjektive
\item[(c)]$A$ er verdiområdet, $\{1, 2, 4\}$ er bildemengden
\item[(d)]Om jeg forstår oppgaven riktig tror jeg $g(f(2))$ 4 og $f(g(2))$ er ikke definert.
\item[(e)]En funksjon kan ikke sende et element fra definisjonsområdet til to ulike elementer i verdiområdet. I dette tilfellet bryter de to tuplene $\langle R,Q\rangle$ og $\langle R,R\rangle$ dette, og derfor er ikke uttrykket definert som en funksjon
\item[(f)] I følge definisjonen av en funksjon må ethvert element i definisjonsområdet sendes til et element i verdiområdet. I dette tilfellet er $4\in definisjonsmengden$ men sendes ikke til et element i verdimengden.
\end{description}

\section*{Oppgave 10.8}
\begin{description}
\item[(a)]Den refleksive tillukningen av R er $R \cup \{\langle 1,1\rangle,\langle 2,2\rangle,\langle 3,3\rangle,\langle a,a\rangle,\langle b,b\rangle\}$
\item[(b)]Den symmetriske tillukningen av R er $R \cup \{\langle a,1\rangle\}$
\item[(c)]Den transitive tillukningen av R er $R \cup \{\langle 2,2\rangle,\langle 3,3\rangle\}$
\end{description}

%Her er noen flere uttrykk:
%\begin{align*}
%	P &= Q \cup S \\
%	S &= Q \cap R \\
%	S &\subseteq R \\
%	R &= F \setminus R \\ 
%	R &= \emptyset
%\end{align*}

%Her er et uttrykk: $P = \{a, b, c \}$.

\end{document}

