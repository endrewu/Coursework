% Kompiler med "pdflatex latex.tex"
% LaTeX på Wikibooks: http://en.wikibooks.org/wiki/LaTeX/
% Finn symboler ved å tegne: http://detexify.kirelabs.org/classify.html

\documentclass[12pt,norsk,a4paper]{article}

\usepackage[T1]{fontenc}
\usepackage[utf8]{inputenc}
\usepackage[norsk]{babel}
\usepackage{amsmath}
\usepackage{parskip}
\usepackage{fixltx2e}

\title{Innlevering 8 INF1080}
\author{Endre Wullum\\ \texttt{endrewu@ulrik.uio.no}}

\begin{document}

\maketitle

\section*{Oppgave 13.4}
Vis ved induksjon at påstanden $n\textsuperscript{3} - n$ er delelig med 3 for alle naturlige tall $n$.

Vi erstatter $n$ med tallet 0, og får $0\textsuperscript{3} - 0 = 0$ som er delelig med 3.

Vi antar at påstanden holder for $n = k$, induksjonshypotesen, da må det også holde for $n = k + 1$

Fordi induksjonshypotesen sier at $k\textsuperscript{3} - k$ er delelig med 3 må det være et tall m slik at $k\textsuperscript{3} - k = 3m$

$(k+1)\textsuperscript{3} - (k+1) = k\textsuperscript{3} + 3k\textsuperscript{2} + 2k = (k\textsuperscript{3} - k) + 3k\textsuperscript{2} + 3k = 3m + 3k\textsuperscript{2} + 3k$ ved induksjonshypotesen, som er delelig med 3.

Vi kan konkludere med at påstanden er sann for $n=k+1$. Ved induksjon følger det at påstanden er sann for alle naturlige tall.

\section*{Oppgave 14.6}
\begin{description}
\item[(c)]
\begin{enumerate}
\item $f(f(b)) = b$
\item Induksjon
\item induksjonshypotesen
\item $f(bx) = f(b)x$
\item $f(f(b)x)$
\item punkt 2
\item $f(b)0$
\item punkt 3
\item induksjon
\item påstanden er sann
\end{enumerate}
\end{description}

%Her er noen flere uttrykk:
%\begin{align*}
%	P &= Q \cup S \\
%	S &= Q \cap R \\
%	S &\subseteq R \\
%	R &= F \setminus R \\ 
%	R &= \emptyset
%\end{align*}

%Her er et uttrykk: $P = \{a, b, c \}$.

\end{document}

