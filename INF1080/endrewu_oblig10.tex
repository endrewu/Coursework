% Kompiler med "pdflatex latex.tex"
% LaTeX på Wikibooks: http://en.wikibooks.org/wiki/LaTeX/
% Finn symboler ved å tegne: http://detexify.kirelabs.org/classify.html

\documentclass[12pt,norsk,a4paper]{article}

\usepackage[T1]{fontenc}
\usepackage[utf8]{inputenc}
\usepackage[norsk]{babel}
\usepackage{amsmath}
\usepackage{parskip}
\usepackage{fixltx2e}

\title{Innlevering 10 INF1080}
\author{Endre Wullum\\ \texttt{endrewu@ulrik.uio.no}}

\begin{document}

\maketitle

\section*{Oppgave 17.9}
Denne oppgaven forstår jeg ikke helt. Variablene får det til å gå litt i surr for meg og jeg finner ikke noen gode eksempler på lignende oppgaver i forelesningsnotatene. Men her er et forsøk.
\begin{description}
\item[(a)]La relasjonssymbolet tolkes slik at $R\textsuperscript{M} = \{1, 2\}$. Da er den sann fordi $P(\overline{1}, \overline{1})$, $P(\overline{1}, \overline{2})$, $P(\overline{2}, \overline{1})$ og $P(\overline{2}, \overline{2})$ er sanne, og de er sanne fordi $1, 2 \in R\textsuperscript{M}$
\item[(b)]La relasjonssymbolet tolkes slik at $R\textsuperscript{M} = \{1, 2\}$. Da er den sann fordi $P(\overline{1}, \overline{1})$, $P(\overline{1}, \overline{2})$, $P(\overline{2}, \overline{1})$ og $P(\overline{2}, \overline{2})$ er sanne, og de er sanne fordi $1, 2 \in R\textsuperscript{M}$
\end{description}
\section*{Oppgave 17.12}
\begin{description}
\item[(a)]La M være en vilkårlig modell, og D domenet til M.

For å vise at $M \models (F \rightarrow G)$ sann er det tilstrekkelig å vise at hvis M gjør F sann, så gjør M også G sann.

Anta at M gjør $Pa \lor Pb$ er sann.

Da må vi vise at M gjør $\exists xPx$ sann.

Fordi M gjør $Pa \lor Pb$ sann må M også gjøre $\exists xPx$ sann.

Altså kan vi konkludere at $Pa \lor Pb \rightarrow \exists xPx$ er gyldig.
\item[(b)]La M være en vilkårlig modell, og D domenet til M.

For å vise at $M \models (F \rightarrow G)$ er sann er det tilstrekkelig å vise at hvis M gjør F sann så gjør M også G sann.

Anta at M gjør $\forall xPx$ er sann.

Da må vi vise at M gjør $Pa \land Pb$ sann.

Fordi M gjør $\forall xPx$ sann må M også gjøre $Pa \land Pb$ sann.

Altså kan vi konkludere at $\forall xPx \rightarrow Pa \land Pb$ er gyldig.
\end{description}


%Her er noen flere uttrykk:
%\begin{align*}
%	P &= Q \cup S \\
%	S &= Q \cap R \\
%	S &\subseteq R \\
%	R &= F \setminus R \\ 
%	R &= \emptyset
%\end{align*}

%Her er et uttrykk: $P = \{a, b, c \}$.

\end{document}

