% Kompiler med "pdflatex latex.tex"
% LaTeX på Wikibooks: http://en.wikibooks.org/wiki/LaTeX/
% Finn symboler ved å tegne: http://detexify.kirelabs.org/classify.html

\documentclass[12pt,norsk,a4paper]{article}

\usepackage[T1]{fontenc}
\usepackage[utf8]{inputenc}
\usepackage[norsk]{babel}
\usepackage{amsmath}
\usepackage{parskip}
\usepackage{fixltx2e}

\title{Innlevering 11 INF1080}
\author{Endre Wullum\\ \texttt{endrewu@ulrik.uio.no}}

\begin{document}

\maketitle

\section*{Oppgave 19.12}
\begin{description}
\item[(a)]$\{4, 5\}$
\item[(b)]$\{\emptyset\{1\}, \{4\}, \{1,4\}\}$
\item[(c)]$\{\{a,b,c,d\},\{e,f\}\}$
\item[(d)]$\{\{1,2,3\},\{4\}\}$
\item[(e)]Fordi elementet 2 er en del av begge delmengdene. I henhold til definisjonen av en partisjon må snittet mellom alle delmengdene være den tomme mengden.
\item[(f)]Fordi en potensmengde er definert slik at potensmengden av A er mengden av alle delmengder av A må alltid $X \in \mathcal{P}$ fordi $X \subseteq X$.
\end{description}

\section*{Oppgave 19.13}
Vi vet at for alle $x \in S$ og alle $y \in S$ så er det slik at xRy.

Fordi relasjonen er symmetrisk må det derfor også være slik at yRx.

Fordi vi også vet at relasjonen er transitiv må derfor xRx og yRy.

Nå som vi har sett at relasjonen er transitiv, symmetrisk og refleksiv kan vi derfor konkludere med at den er en ekvivalensrelasjon.

%Her er noen flere uttrykk:
%\begin{align*}
%	P &= Q \cup S \\
%	S &= Q \cap R \\
%	S &\subseteq R \\
%	R &= F \setminus R \\ 
%	R &= \emptyset
%\end{align*}

%Her er et uttrykk: $P = \{a, b, c \}$.

\end{document}

