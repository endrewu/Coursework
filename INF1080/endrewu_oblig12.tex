% Kompiler med "pdflatex latex.tex"
% LaTeX på Wikibooks: http://en.wikibooks.org/wiki/LaTeX/
% Finn symboler ved å tegne: http://detexify.kirelabs.org/classify.html

\documentclass[12pt,norsk,a4paper]{article}

\usepackage[T1]{fontenc}
\usepackage[utf8]{inputenc}
\usepackage[norsk]{babel}
\usepackage{amsmath}
\usepackage{parskip}
\usepackage{fixltx2e}

\title{Innlevering 12 INF1080}
\author{Endre Wullum\\ \texttt{endrewu@ulrik.uio.no}}

\begin{document}

\maketitle

\section*{Oppgave 20.15}
\begin{description}
\item[(a)]Relasjonen er ikke en ekvivalensrelasjon da $\langle e,e\rangle$ ikke er et element i mengden. Altså er den ikke refleksiv.
\item[(b)][a] = $\{a, b\}$
\item[(c)][a] = $\{a\}$
\item[(d)]Relasjonen er ikke en ekvivalensrelasjon da den inneholder tuplene $\langle b,c\rangle$ og $\langle b,c\rangle$ men ikke $\langle c,a\rangle$ $\langle c,b\rangle$. Altså er den ikke symmetrisk.
\item[(e)][a] = $\{a\}$
\item[(f)][a] = $\{a,c\}$
\end{description}

\section*{Oppgave 20.16}
\begin{description}
\item[(a)]Per definisjon er [0] en delmengde av E. Siden ~ er refleksiv må 0 ~ 0, følgelig at $0 \in [0]$. [0] kan altså ikke være den tomme mengden.
\item[(b)]Anta, som gitt, at $x \in E$ og $y \in E$. Definisjonsmessig er også $0 \in E$, x ~ 0 og y ~ 0. Fordi ~ er symmetrisk og transitiv må derfor x ~ y.
\item[(c)] La R være en relasjon på ekvivalensklassene [x] og [y]. Definisjonsmessig vet vi at $x \in [x]$ og at $y \in [y]$. Da vet vi at xRx og yRy, altså at relasjonen er refleksiv. 
\end{description}

Fordi ekvivalensklassene er like og vi vet at alle elementene i en ekvivalensklasse er relatert til hverandre må xRy og yRx, altså må relasjonen være symmetrisk. Fordi alle elementene er relatert til hverandre må den intuitivt også være transitiv.

Fordi relasjonen er refleksiv, symmetrisk og transitiv er den en ekvivalensrelasjon. Derfor må det være slik at hvis [x] = [y] så x ~ y.

\section*{Oppgave 21.16}
Det er $5^5$, altså 3125 funksjoner fra S til S.
Av disse er 5!, altså 120 bijektive.

%Her er noen flere uttrykk:
%\begin{align*}
%	P &= Q \cup S \\
%	S &= Q \cap R \\
%	S &\subseteq R \\
%	R &= F \setminus R \\ 
%	R &= \emptyset
%\end{align*}

%Her er et uttrykk: $P = \{a, b, c \}$.

\end{document}

